%!TEX root = volumeFinal.tex 

\chapter{\label{concl}Conclusões}

\frm{Eu esperava alguma conclusão mais profunda sobre o AHTN, sua aplicabilidade, a dificuldade de modelar domínios? Se tu for falar do teu aprendizado, fale também de como foi implementar e fazer a avaliação objetiva das coisas.}
A realização deste trabalho foi de grande valor, pois me proporcionou grande aprendizado sobre agentes, técnicas de busca, planejamento, e jogos RTS.
A implementação do MicroRTS está bem estruturada, por isso não houve dificuldades no entendimento da plataforma.
O planejador JSHOP2 é de fácil acesso, e sua documentação é bem rica.
Mas os exemplos de descrição do domínio e problema disponíveis com o planejador não cobrem todas as funcionalidades.
Por essa razão a aplicação do planejador foi a grande dificuldade deste trabalho.

A implementação do algoritmo de mostrou valida em relação as outra técnicas presentes no MicroRTS.
Em todos os casos, o tempo de geração das ações é maior no algoritmo de AHTN do que nas outras técnicas.
Por exemplo, a técnica de WorkerRush, que não perdeu em nenhum dos teste, leva 1 milissegundo para determinar as suas ações.
Além do tempo de geração das ações, os conhecimentos de domínio não avaliam todas as possibilidades relevantes do jogo, deixando o algoritmo limitado em relação as ações.

O domínio 2 obtém melhor resultado do que o domínio 1.
O que já era esperado devido ao treinamento de mais unidades de ataque.
Um fator interessante na analise dos resultados foi o lado do jogo, pois ele influenciou bastante os resultados.
Algumas técnicas do MicroRTS não realizavam as mesmas ações do lado azul, isso faz com que a vantagem do algoritmo de AHTN no lado vermelho seja alta.


Para trabalhos futuros pretende-se modelar outros conhecimentos de domínio, com o intuito de aumentar as possibilidades de jogadas do algoritmo. 
Além disso, pretende-se realizar o acoplamento do algoritmo de AHTN com alguma técnica de \textit{Machine Learning}.

