%!TEX root = volumeFinal.tex 
\chapter{\label{chap:busca}Busca}

\section{Busca adversaria}

A busca adversaria é utilizada para ambientes competitivos, como nos jogos. Como em um jogo o jogador, preferencialmente, não informa suas jogadas previamente, o ambiente se torna imprevisível, e com isso os objetivos dos jogadores entram em conflito, ambos estão em busca da vitoria. Como solução para esse problema é preciso gerar uma solução de contingencia para tentar antecipar as jogadas do adversário \cite{intelligence2003modern}. 

Para explicar como resolver esse problema, primeiro é preciso considerar um jogo com dois jogadores, um é chamado de MAX e o outro de MIN. O jogador MAX começa o jogo e o jogo e então é alternado uma jogada de MIN e uma de MAX até o final do jogo. Ao final do jogo, quem vence obtém uma recompensa positiva e quem perde uma negativa. Um jogo pode ser formalizado como \cite{intelligence2003modern}:

\begin{itemize}
	\item $S_{0}$ - O estado inicial, que especifica como o jogo se configura no inicio.
	\item Jogadores(s) -  Define qual jogador tem o movimento no estado.
	\item Ações(s) - Conjunto das ações possíveis em um estado.
	\item Resultado(s, a) - Um modelo de transição, que define o resultado da ação a aplicada ao estado s.
	\item Terminal(s) - Verifica se o estado é um estado onde o jogo terminou.
	\item Utilidade(s,p) - Define qual é o valor numérico para o jogo quando atingir um estado s terminal por um jogador p. 
\end{itemize}

O estado inicial, as ações e os resultados definem a arvore das jogadas para o jogo. A arvore representa em cada nodo um estado do jogo e cada ligação com os níveis de baixo são os estados resultantes após a execução de cada ação possíveis para o estado. A alternância entre as jogadas de MAX e MIN até chegar as folhas da arvore que correspondem aos estados terminais. Como o ponto de vista é do MAX, o valor de cada nodo folha representa o valor de utilidade para o MAX, e os maiores valores representam bons resultados para o MAX e ruins para o MIN. Com isso o caminho resultante indica que aquela ação será a melhor ação para o estado atual. 

\msr[inline]{colocar uma figura de uma game tree?}

Este tipo de busca leva em consideração que o jogador adversário sempre realizará a jogada que mais lhe beneficiará. Um algoritmo que utiliza deste recurso é o algoritmo \textit{minimax search}.
