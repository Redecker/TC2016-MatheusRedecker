%!TEX root = volumeFinal.tex 

\chapter{\label{chap:impl}Implementação}

\section{Modelagem do domínio}

Antes de realizar a implementação do algoritmo de AHTN, é preciso criar um domínio de planejamento. 
O domínio de planejamento serve para que o JSHOP2 consiga gerar os planos que serão usados pelo algoritmo.

Antes de iniciar a criação do domínio, foi preciso definir qual estratégia o domínio iria adotar.
O primeiro domínio foi pensado em uma situação de jogo, onde o jogador quer treinar uma unidade de ataque para destruir o seu adversário.
Sendo assim, o jogador precisar ter um \textit{worker}, para coletar recursos, um quartel para construir unidades de ataque, e uma base para conseguir guardar seus recursos. 
Após ter um quartel, o jogador treina uma unidade de ataque, e manda ela atacar.
A unidade \textit{ranged} foi a unidade de ataque escolhida para ser treinada.
Apenas quando o \textit{ranged} morre que outro é criado.
Este domínio precisa que o jogo comece com uma base, pois a base é necessária para treinar \textit{workers} e para guardar os recursos coletados.  
O domínio foi pensado para ter apenas um \textit{worker} para realizar a coleta de recursos, e um quartel para treinar as unidades de ataque.

Na camada de abstração do MicroRTS temos movimentos que podem ser aplicados dentro do jogo. 
Cada movimento foi transcrito como um operador.
Os métodos são usados para terminar quando esses operadores podem ser utilizados.



Duas modelagens de domínio 
dominio pensado para criar unidade e atacar
estrategia 1
apendice A

\section{Heurística}

Quando o minimax chega ao fim, deve decidir como mostrar o valor de utilidade pro estado
duas heuristicas
	-levando em conta um peso para cada tropa e edificação
	-levando em conta as suas coisas menos a do outro

\section{Implementação}

algoritmo
	-estado do jogo
	-gera o plano e gera denovo a cada vez
	-

jshop dentro do java
	-gera as classes uma vez e as traduz dentro do algoritmo na mão
	-tradução do problema de planejamento
	
