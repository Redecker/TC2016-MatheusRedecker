%!TEX root = volumeFinal.tex 

\chapter{\label{chap:impl}Implementação}

\section{Modelagem do domínio}

Antes de realizar a implementação do algoritmo de AHTN, é preciso criar um domínio de planejamento. 
O domínio de planejamento serve para que o JSHOP2 consiga gerar os planos que serão usados pelo algoritmo.

Antes de iniciar a modelagem do domínio, é preciso definir qual estratégia o domínio vai seguir.
O primeiro domínio é pensado em uma situação de jogo, onde o jogador quer treinar uma unidade de ataque para destruir o seu adversário.
Sendo assim, o jogador precisar ter um \textit{worker}, para coletar recursos, um quartel para construir unidades de ataque, e uma base para conseguir guardar seus recursos e criar o seu \textit{worker}. 
Após ter um quartel, o jogador treina uma unidade de ataque, e manda ela atacar.
A unidade \textit{ranged} foi a unidade de ataque escolhida para ser treinada.
Apenas quando a unidade \textit{ranged} é destruída que outra unidade é criada.
Este domínio precisa que o jogo comece com uma base, pois a base é necessária para treinar \textit{workers} e para guardar os recursos coletados.  
O domínio foi pensado para ter apenas um \textit{worker} para realizar a coleta de recursos, e um quartel para treinar as unidades de ataque.

A modelagem do domínio começa pela descrição dos operadores.
Cada ação que pode ser executada dentro do jogo é descrita como um operador no domínio.
Por exemplo, as ações de construir um quartel, e coletar recursos, são os operadores $!buildbarrack$ e $!getresource$ respectivamente.
Já os métodos servem para determinar ações de mais alto nível, e devem respeitar a estratégia definida.
Por exemplo, a estratégia define que apenas um quartel é construído, então o método deve ter uma precondição para verificar se não existe outro quartel no jogo, caso não tenha, ainda é preciso checar se há recursos suficientes para construir. 
Cada método pode chamar outros métodos.
Isso acontece pois pode ser necessário realizar outra ação antes de conseguir realizar a ação do método.
Por exemplo, é necessário um quartel para treinar uma unidade de ataque, quando o método de treinar unidade é chamado, ele verifica que não há um quartel, e ele chama o método de construir o quartel.

Para ganhar o jogo é preciso atacar o adversário, por essa razão o método utilizado como objetivo no problema de planejamento é o $ataqueranged$.
Esse método desencadeia todas as outras chamadas de método dependendo do estado do ambiente.
No Apêndice~\ref{ap:estra1} está a descrição deste domínio que é chamado de Estratégia 1.

\section{Heurística}

O algoritmo de AHTN requer uma função de avaliação para determinar o valor de utilidade para cada estado do jogo.
Essa função utiliza uma heurística para determinar esse valor de utilidade.
A primeira heurística criada leva em consideração apenas as unidades do jogador.
Quanto maior o número de unidades disponíveis no ambiente melhor a função de avaliação do estado.
As unidades são ponderadas pelo custo de treinamento ou de construção.
Por exemplo, um quartel precisa de 5 unidades de recurso para ser construído, enquanto um \textit{worker} necessita só de 1 para ser treinado, por isso o quartel acaba tendo peso 5, e o \textit{worker} peso 1. 
Essa heurística tem um problema, pois o jogador adversário pode estar com o dobro de unidades que isso não é levado em consideração, e nesse cenário o jogador adversário estaria na frente. 
Essa heurística é chamada de heurística 1.

Para resolver o problema das unidades do inimigo não serem levadas em consideração.
A segunda heurística utiliza a mesma maneira de calcular da primeira heurística, mas também calcula esse valor para as unidades do adversário. 
Com os dois valores, o valor das suas unidades é subtraído do valor das unidades do adversário. 
Assim em um cenário onde o jogo está com mais tropas a favor, o valor de utilidade é positivo, e caso o contrario negativo.
Essa heurística é chamada de heurística 2.

\section{Implementação}

O primeiro passo da implementação é conseguir gerar planos dentro do MicroRTS.

jshop dentro do java
-gera as classes uma vez e as traduz dentro do algoritmo na mão
-tradução do problema de planejamento


algoritmo
	-estado do jogo
	-gera o plano e gera denovo a cada vez
	-

	
