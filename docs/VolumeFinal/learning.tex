%!TEX root = volumeFinal.tex 
\chapter{\label{chap:aprendizado}Aprendizado}
 
Para os humanos o aprendizado ocorre durante toda a vida. 
O aprendizado é o ato de adquirir novos conhecimentos, ou modificar conhecimentos já existentes ou ainda adquirir uma experiencia por repetição do ato de forma incorreta. 
Aprendizado pode variar de adquirir conhecimento de tarefas simples, como decorando um numero de telefone, até tarefas mais complicadas, como a formulação de novas teorias \cite{intelligence2003modern}. 
% Não precisa colocar quebra de parágrafo em todo o lado!

\section{Aprendizado de Máquina} 

A área na computação que estudo esse aprendizado de forma computacional é o aprendizado de maquina, melhor conhecida como \textit{machine learning}. A definição de aprendizado de maquina proposta por Tom Mitchell \cite{Mitchell1997ML} é a seguinte:

\begin{quote}
	Definição: Um programa de computador é dito que aprende de uma experiencia E com relação a alguma classe de tarefas T, e medida de performance P, se essa performance sobre as tarefas em T, medida por P, melhora com a experiencia E.
\end{quote}

Essa definição mostra que o sistema aprimora seu conjunto de tarefas T com uma performance P através de experiencias E. Ou seja, um sistema baseado em aprendizado de maquina deve, através de experiencias, ter um ganho nas informações para solucionar os seus problemas. Para começar a resolver um problema utilizando aprendizado de maquina é preciso escolher qual experiencia será aprendida pelo sistema \cite{Mitchell1997ML}. Para isso existem algumas técnicas que tratam aprendizado de maquina com objetivos diferentes \cite{intelligence2003modern}. Alguma das técnicas são: 
\begin{itemize}
	\item Aprendizado supervisionado: Aprender através de algum conjunto de exemplos a realizar a classificação de algum problema. Cada problema é mapeado para uma saída. 
	\item Aprendizado não supervisionado: Aprender através das observações, algum pedrão ou regularidade, para classificar em grupos os problemas.
	\item Aprendizado por reforço: Aprender através das execuções, bem ou mal sucedidas. Aprende quais ações são melhores de ser executadas.
\end{itemize}

Como podemos ver cada tipo de aprendizado é utilizado para uma aplicação especifica...

\section{Aprendizado por Reforço}

O aprendizado por reforço também é conhecido como \textit{reinforcement learning}. O objetivo deste aprendizado é usar as recompensas obtidas nas observações para aprender uma politica do ambiente \cite{intelligence2003modern}. Para que isso aconteça, cada estado contém uma utilidade, que mostra o quão desejável é este estado, pode ser um desejo bom ou um ruim. Com essa informação é possível determinar, dado as ações disponíveis no estado, qual ação o sistema deve escolher para seguir a execução. 
