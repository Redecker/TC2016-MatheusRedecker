%!TEX root = proposta.tex 
%% Matheus renomeia "exemplo.tex" para um nome mais descritivo (e muda a linha acima)
\chapter{\label{chap:intro}Introdução}

Inteligencia artificial(IA) é uma área em ciência da computação que tem como objetivo fazer com que o computador seja capaz de realizar tarefas que precisam ser pensadas, como é feito pelas pessoas.  
A IA possui algumas áreas de aplicação, tais como: aprendizado, planejamento, jogos, e mineração de dados entre outras. 

%A utilização de IA para jogos começou simples, com a aplicação de técnicas para jogos como xadrez e jogo da velha, mas atualmente é difícil encontrar um jogo que não utilize alguma técnica de IA.\frm{Quem disse isto? Citação ou não existiu.} 
As técnicas de IA utilizadas nos jogos são necessárias para conseguir uma melhor interação com o jogador, tornando o jogo mais real e assim prendendo a atenção do jogador \cite{millington2009artificial}. 
As técnicas utilizadas nos jogos, geralmente, são mais simples do que as que utilizadas no meio acadêmico, pelo fato de que o tempo de resposta dos algoritmos é superior ao tempo que se tem para tomar uma ação ótima dentro do jogo \cite{intelligence2003modern}. 
Nos jogos as reações devem ser quase que imediatas, para isso técnicas que tentam explorar todo o espaço de estados do jogo se tornam inviáveis para jogos mais complexos.
Por exemplo, no xadrez a quantidade aproximada de estados possíveis é de $10^{40}$, isso mostra que o poder de processamento para gerar, de maneira rápida, uma ação precisa ser alto \cite{millington2009artificial}. Então é difícil conseguir gerar uma ação ótima, em alguns casos são gerados ações sub ótimas para que o tempo de resposta não seja muito alto \cite{intelligence2003modern}.     %\frm{Tá, agora tu tens que concluir alguma coisa... Então é complicado gerar uma ação ótima, e daí?}


%\frm[inline]{De repente falar de busca adversária antes, ou algo assim. Eu sei que tu tens que conectar estas idéias.}
A busca é utilizada, dendo da IA, para achar a possível sequencia de ações que resolve um problema, considerando varias possibilidades de sequencia dessas ações. Os algoritmos de busca se diferenciam entre si na forma de escolher qual o próximo estado na busca pelo objetivo. Já a busca adversária é utilizada para a resolução de problemas de busca em modo competitivo. A busca adversária pressupõe que sempre o oponente irá realizar sempre a jogada que mais lhe beneficia, isso nem sempre acontece, seja porque o jogador é iniciante ou comete um erro. O ser humano consegue raciocinar para decidir as ações, mas nem sempre ele é perfeito nas escolhas das ações, ele consegue ser muito bom, mas o modo de pensar não o torna perfeito sempre. Planejamento é uma área da IA que busca a geração de planos de forma automática, parecido com a busca, utilizando técnicas para buscar a geração de um plano que satisfaça um objetivo. A utilização de técnicas de planejamento em jogos é uma tarefa difícil devido a sua grande quantidade de ações possíveis. Esse gênero de jogo possui um fator de ramificação muito grande, e cresce exponencialmente, com isso aplicar algoritmos de planejamento se torna uma tarefa não trivial \cite{intelligence2003modern}. 


%\frm{Tu conectaste as idéias melhor no abstract, aqui tu terias que introduzir algo no sentido de que: para mitigar as limitações de eficiência computacional de abordagens tradicionais de raciocínio em jogos, X e Y propuseram o AHTN}
Na busca de mitigar as limitações de eficiência computacional de abordagens tradicionais de raciocínio em jogos, Santiago Ontañón e Michael Buro propuseram o algoritmo chamado \textit{Adversarial Hierarchical Task Network (AHTN)} \cite{ontanon2015adversarial}. Neste algoritmo são combinadas técnicas de HTN com o algoritmo de busca adversaria \textit{minimax search}. 

Propomos a utilização do algoritmo AHTN combinado com uma técnica de aprendizado por reforço em um jogo de estrategia em tempo real combinando uma técnica de aprendizado por reforço. Com este trabalho pretendemos mostrar que o algoritmo de AHTN apresenta melhores resultados quando aplicado junto com técnicas de aprendizado por reforço.



%Para resolver esse problema existe uma proposta de solução utilizando o algoritmo de AHTN \cite{ontanon2015adversarial}.\frm{Jargão} Para melhor o desempenho desse algoritmo, proponho uma união desse algoritmo com um algoritmo de aprendizado de maquina através da ferramenta WEKA. \frm[inline]{A ferramenta que tu usa, neste ponto, é irrelevante. Seja simples e direto para dizer o que tu queres fazer.}

%\begin{itemize}
%\item introdução de IA
%\item linkar IA a jogos
%\item dificuldades de aplicação em jogos
%\item motivação 
%\end{itemize}