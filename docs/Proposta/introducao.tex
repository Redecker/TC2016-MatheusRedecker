%!TEX root = proposta.tex 
%% Matheus renomeia "exemplo.tex" para um nome mais descritivo (e muda a linha acima)
\chapter{\label{chap:intro}Introdução}


\todo[color=red,author=Felipe,inline]{
Perguntas que teu texto tem que responder:
\\Qual é o problema?
\\Por que ele é interessante?
\\Como tu pretende resolver ele?
}

Inteligencia artificial é uma área dentro da Ciência da computação. Essa área tem como objetivo o desenvolvimento de técnicas com intuito de representar comportamentos inteligentes dentro de um computador. A inteligencia artificial possui algumas áreas de aplicação: 
\begin{itemize}
	\item Aprendizado
	\item Planejamento
	\item Jogos
\end{itemize}

Jogos eletrônicos são muito populares, não só entre os jovens, principalmente pela grande quantidade de gêneros, existem jogos de ação, aventura, esportes, estrategia, entre outros. \\
Dentro dos jogos de estratégia há uma subseção que se chama jogos de estratégia em tempo real, neles os jogadores estão se enfrentando no mesmo momento, como o nome já diz. Em alguns desses jogos há BOTs(jogadores que simulam um jogador real) e é preciso alguma inteligencia para esses BOTs conseguirem levar graça ao jogador real, para isso é utilizado algoritmos de IA. \\
A utilização de técnicas de planejamento em jogos em tempo real é uma tarefa difícil devido a sua grande quantidade de ações possíveis. Esse gênero de jogo possui um fator de ramificação muito grande, e cresce exponencialmente, com isso aplicar algoritmos de planejamento se torna uma tarefa não trivial. \\
Em algoritmos de aprendizado de maquina esse tipo de problema não acontece, pois os algoritmos podem ser construídos para aprender com o ambiente através das execuções,  mas a dificuldade vem do fato que é difícil modelar todas as variáveis que devem ser utilizadas nesses algoritmos a fim de cobrir completamente o comportamento do sistema. \\
Para resolver esse problema existe uma proposta de solução utilizando o algoritmo de AHTN \cite{ontanon2015adversarial}. Para melhor o desempenho desse algoritmo, proponho uma união desse algoritmo com um algoritmo de aprendizado de maquina através da ferramenta WEKA. \\

%\begin{itemize}
%\item introdução de IA
%\item linkar IA a jogos
%\item dificuldades de aplicação em jogos
%\item motivação 
%\end{itemize}