%!TEX root = proposta.tex 
%% Matheus renomeia "exemplo.tex" para um nome mais descritivo (e muda a linha acima)
\chapter{\label{chap:intro}Introdução}


\todo[color=red,author=Felipe,inline]{
Perguntas que teu texto tem que responder:
\\Qual é o problema?
\\Por que ele é interessante?
\\Como tu pretende resolver ele?
}

Inteligencia artificial é uma em ciência da computação que tem como objetivo o desenvolvimento de técnicas para representar e executar comportamentos inteligentes em um computador. 
A inteligencia artificial possui algumas áreas de aplicação, tais como: %\frm{Isto é uma lista exaustiva?} 
\begin{itemize}
	\item Aprendizado;
	\item Planejamento;
	\item Jogos;
	\item Mineração de dados;
	\item Escalonamento de tarefas.
\end{itemize}

\frm[inline]{Tu estás tentando explicar em profundidade os tipos de jogos, mas talvez seja melhor tu simplesmente colocar o problema mais genérico de programar As para jogos, depois te preocupa em detalhar os tipos de jogos e tipos de técnicas (isto é a parte final  da introdução)}

Em jogos

A área de jogos eletrônicos é muito popular %, não só entre os jovens\frm{Se tu tens que falar ``nos jovens'', tu parece um velho, hehe},
, principalmente pela grande quantidade de gêneros, existem jogos de ação, aventura, esportes, estrategia, entre outros. \frm{Evitar parágrafos de sentenças solitárias como este}


Dentro dos jogos de estratégia há uma subgênero que se chama jogos de estratégia em tempo real, neles os jogadores estão se enfrentando no mesmo momento, como o nome já diz. 
Em alguns desses jogos há BOTs(jogadores que simulam um jogador real) e é preciso alguma inteligencia para esses BOTs conseguirem levar graça ao jogador real, para isso é utilizado algoritmos de IA. 

\frm[inline]{Estás entrando direto em jargão (planejamento), esta introdução tem que ser legível sem falar do jargão.}
A utilização de técnicas de planejamento em jogos em tempo real\frm{O que é planejamento em tempo real?} é uma tarefa difícil devido a sua grande quantidade de ações possíveis. 
Esse gênero de jogo possui um fator de ramificação muito grande, e cresce exponencialmente, com isso aplicar algoritmos de planejamento se torna uma tarefa não trivial. 


Em algoritmos de aprendizado de maquina esse tipo de problema não acontece, pois os algoritmos podem ser construídos para aprender com o ambiente através das execuções,  mas a dificuldade vem do fato que é difícil modelar todas as variáveis que devem ser utilizadas nesses algoritmos a fim de cobrir completamente o comportamento do sistema. \frm{Não são só estas as dificuldades, de novo, não entra no aprendizado de máquina ainda.}
\frm[inline]{Aqui tu só queres oferecer pro leitor o fato de que tu vais usar técnicas de planejamento (mas antes de falar de planejamento, dizes em duas sentenças o que é isto), e busca adversária para resolver um problema muito difícil.}


Para resolver esse problema existe uma proposta de solução utilizando o algoritmo de AHTN \cite{ontanon2015adversarial}.\frm{Jargão} Para melhor o desempenho desse algoritmo, proponho uma união desse algoritmo com um algoritmo de aprendizado de maquina através da ferramenta WEKA. \frm[inline]{A ferramenta que tu usa, neste ponto, é irrelevante. Seja simples e direto para dizer o que tu queres fazer.}

%\begin{itemize}
%\item introdução de IA
%\item linkar IA a jogos
%\item dificuldades de aplicação em jogos
%\item motivação 
%\end{itemize}