%!TEX root = proposta.tex 
%% Matheus renomeia "exemplo.tex" para um nome mais descritivo (e muda a linha acima)
\chapter{\label{chap:conte}Contexto}

\todo[color=red,author=Felipe,inline]{Ok, aqui está bem genérico, agora começa a projetar parágrafo por parágrafo do material que tu já leste (bullets para cada parágrafo), e me manda depois que tu completares o texto. Deixa as bullets no topo para eu entender o teu raciocínio}

\cite{meneguzzi2015planning}
\cite{ontanon2015adversarial}
\cite{ontanon2012experiments}
\cite{intelligence2003modern}
\cite{buro2003real}
\cite{hogg2010learning} \msr{apenas para aparecem as referencias}

\section{Planejamento} 
-Overview de planejamento \cite{ghallab2004automated}\\
**O que é em geral \\
**o que é em computação \\
**formas de planejaento\\
**Motivação \\

-definição \msr{acho que ficou desnecessario esse topico}\\

-conceitos \cite{ghallab2004automated}\\
**modelos(assumptions) \\

-metodos \\
*classico \\
**o que é \\
** objetivos \\
** complexidade \\
** limitações \\
*neoclassico
** o que é \\
** tipos \\

-aplicação no mundo real\\
**como se aplica \\
**exemplo de problema \\

\subsection{HTN} 
-como se encaixa em planejamento \\ 
** linkar com planejamento classico\\
** difereças do planejamento classico\\
-definição \\
** o que é\\ 
** decomposição de tarefas \\
** subtask \\
** goals \\
** modelo formal \\
**diferença entre SPN e total order SPN \\
** problemas \\

-conceitos \msr{acho que ficou desnecessario esse topico}\\

\subsection{AHTN} 
-como se encaixa em planejamento e HTN \\
**o que é (HTN + game search tree)\\
** quem propos\cite{ontanon2015adversarial} \\
-definição \\
** game search tree \\
** como encaixou as duas coisas
** algoritmo \\
-conceitos \msr{acho que este topico ficou desnecessario} \\

\section{Aprendizado} 
-overview de aprendizado
**O que é em geral \\
**o que é em computação \\
**formas de machine learning\\
**Motivação \\

\subsection{Aprendizado de Máquina} 
- overview de aprendizado de maquina \\
**
- definição \\
- conceitos

\subsection{Aprendizado por Reforço} 
\todo[color=red,author=Felipe,inline]{Pode ser desnecessário mesmo. Troca o tom do teu azul, é tri agressivo nos olhos, hehe.}
- como se encaixa dentro de aprendizado de maquina \\
** o que é \\
** tipos de learning(passivo e ativo) \\
* passivo \\
** o que é \\
** como funciona \\
** algoritmos \\
* ativo \\
** o que é \\
** como funciona \\
** algoritmos \\

- definição \msr{acho que este topico ficou desnecessario}\\
- conceitos \msr{acho que este topico ficou desnecessario}

\section{Jogos Real-time Strategy(RTS)} 
Explicação do que é jogos RTS \\
**jogos
** RTS
**jogos rts
ligação deles com atividades do mundo real \\
exemplos de jogos RTS 

\subsection{MicroRTS} 
explicação de como funciona, para que foi feito. 
