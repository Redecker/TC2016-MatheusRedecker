%!TEX root = proposta.tex 
\chapter{\label{chap:obje}Objetivos}

Atualmente, há uma dificuldade na utilização de técnicas de IA para jogos em tempo real. Isso ocorre pelo fato que a IA precisa ser capaz de resolver tarefas, do mundo real, de maneira rápida e satisfatória. Geralmente, o espaço de estados dos jogos é enorme, isso faz com que não se tenha tempo de explorar todas as possibilidades de solução.   \\

%\frm{Seja mais específico, o que é difícil? Faltam técnicas?}

O intuito desse trabalho é mostrar que uma abordagem híbrida com planejamento e aprendizado de maquina pode ser uma alternativa valida para tratar esse gênero de jogo. 
%\frm{Não é bem a leitura de livros, tu vais te basear em um algoritmo específico que tu encontraste}
A leitura de livros será utilizada para adquirir embasamento teórico \cite{ghallab2004automated,intelligence2003modern,Mitchell1997ML} para entender as técnicas utilizadas no algoritmo AHTN \cite{ontanon2015adversarial}. Já a leitura dos artigos ajudará com a análise dos resultados \cite{ontanon2012experiments,hogg2010learning,ontanon2013survey}.
 \\ 
O jogo escolhido como plataforma foi o MicroRTS \cite{ontanon2013combinatorial}, pelo fato de que o jogo é uma simplificação de um jogo em tempo real, e é utilizado para demonstrar outras técnicas de IA . Como o jogo escolhido como plataforma já tem alguns algoritmos de IA implementados %\frm{Tu estás mencionando o jogo o qual tu ainda nem falou!!}
, a comparação com outros algoritmos pode ser considerado algo mais simples. %
%\frm{Talvez mover os objetivos para depois?} 

%\section{Objetivo Geral}

%O objetivo geral deste trabalho é implementar o algoritmo de planejamento AHTN \cite{ontanon2015adversarial} que combina técnicas de HTN com o algoritmo de \textit{minimax serach}. Após implementado, integrar aprendizado por reforço através da ferramenta WEKA\footnote{http://www.cs.waikato.ac.nz/ml/weka/} com o objetivo de melhorar o desempenho do algoritmo. Ao final do trabalho, o objetivo é conseguir comparar resultados obtidos com os resultados já publicados e consolidados \cite{ontanon2007case,ontanon2012experiments,hogg2010learning,buro2003real,ontanon2013survey}.  

\section{Objetivos Específicos}
\label{obj:esp}
\begin{itemize}
\item Estudar algoritmos de HTN.
\item Estudar o algoritmo de AHTN.
\item Estudar algoritmos de aprendizado por reforço.
\item Estudar como unir o algoritmo de AHTN com uma técnica de aprendizado por reforço. %\frm{E o AHTN? Te lembra que usamos aprendizado de máquina em minimax para compensar a falta de tempo para explorar tudo.}
\item Avaliar a aplicabilidade desses algoritmos junto ao jogo escolhido.
\item Definir a estratégia de implementação do algoritmo escolhido.
\item Comparar resultados com outras abordagens.
\end{itemize}



%\frm[inline]{De repente remover este objetivo}
%\section{Objetivo Adicional}

%Caso for possível atingir todos os objetivos propostos na seção \ref{obj:esp}, existe um item adicional %para aplicar a este trabalho:

%\begin{itemize}
%\item Testar o algoritmo em outro jogo RTS, como por exemplo Battle for Wesnoth\footnote{https://www.wesnoth.org/}\frm{Wesnoth não é RTS é um jogo de estratégia por turnos}.
%\end{itemize}
