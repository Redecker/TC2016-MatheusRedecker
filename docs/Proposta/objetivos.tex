%!TEX root = proposta.tex 
\chapter{\label{chap:obje}Objetivos}

Atualmente, há uma dificuldade na utilização de técnicas de IA para jogos em tempo real.\frm{Seja mais específico, o que é difícil? Faltam técnicas?}
O intuito desse trabalho é mostrar que uma abordagem híbrida com planejamento e aprendizado de maquina pode ser uma alternativa valida para tratar esse gênero de jogo. 
\frm{Não é bem a leitura de livros, tu vais te basear em um algoritmo específico que tu encontraste}
A leitura de livros será o ponto de estudo para ganhar embasamento para conseguir aplicar os algoritmos, já a leitura dos artigos ajudará com a análise dos resultados e ainda nas alternativas para resolver os possíveis problemas encontrados. 
Como o jogo escolhido como plataforma já tem alguns algoritmos de IA implementados\frm{Tu estás mencionando o jogo o qual tu ainda nem falou!!}, a comparação com outros algoritmos pode ser considerado algo mais simples. \frm{Talvez mover os objetivos para depois?} 

\section{Objetivo Geral}

O objetivo geral deste trabalho é implementar o algoritmo de planejamento AHTN \cite{ontanon2015adversarial} e ainda integrar aprendizado por reforço através da ferramenta WEKA\footnote{http://www.cs.waikato.ac.nz/ml/weka/} com o objetivo de melhorar o desempenho do algoritmo. Esse trabalho utilizará técnicas que serão estudadas ao longo do semestre.\frm{Isto está vago, tu já identificaste as técnicas, te refira a elas} 
E assim, ao final do trabalho, conseguir comparar resultados obtidos com os resultados já publicados e consolidados \cite{ontanon2007case,ontanon2012experiments,hogg2010learning,buro2003real,ontanon2013survey}.  

\section{Objetivos Específicos}
\label{obj:esp}
\begin{itemize}
\item Estudar algoritmos de planejamento para jogos em tempo real.
\item Estudar algoritmos de aprendizado por reforço.
\item Estudar como unir as técnicas de planejamento com as de aprendizado de máquina.\frm{E o AHTN? Te lembra que usamos aprendizado de máquina em minimax para compensar a falta de tempo para explorar tudo.}
\item Avaliar a aplicabilidade desses algoritmos junto ao jogo escolhido.
\item Definir a estratégia de implementação do algoritmo escolhido.
\item Comparar resultados com outras abordagens.
\end{itemize}

\frm[inline]{De repente remover este objetivo}
\section{Objetivo Adicional}

Caso for possível atingir todos os objetivos propostos na seção \ref{obj:esp}, existe um item adicional para aplicar a este trabalho:

\begin{itemize}
\item Testar o algoritmo em outro jogo RTS, como por exemplo Battle for Wesnoth\footnote{https://www.wesnoth.org/}\frm{Wesnoth não é RTS é um jogo de estratégia por turnos}.
\end{itemize}
