%!TEX root = proposta.tex 
\chapter{\label{chap:ativ}Atividades}

Para está etapa do projeto do Trabalho de Conclusão, foi proposto o plano das atividades apresentado na diagrama de Gantt com a respectiva legenda.

%\frm[inline]{Não é um mapeamento um para um com os objetivos específicos}
\begin{itemize}
\item Task 1 - Escrita do TC I e revisão bibliográfica.
\item Task 2 - Estudar algoritmos de HTN e aprendizado por reforço viáveis para jogos RTS.
%\item Task 3 - Estudar algoritmos de aprendizado de maquina.
\item Task 3 - Avaliar a viabilidade de combinar o algoritmo de AHTN com um algoritmo de aprendizado por reforço através da ferramenta WEKA.
\item Task 4 - Estudar a arquitetura e código fonte do jogo MicroRTS.
\item Task 5 - Projetar a implementação do algoritmo AHTN na plataforma. 
\item Task 6 - Projetar a união do algoritmo AHTN com o de aprendizado por reforço. %\frm{Que algoritmo?}
\end{itemize}
%\frm[inline]{E projetar a implementação? Tu não vais fazer?}

\begin{ganttchart}{1}{14}
	\gantttitle{Proposta 2016/1}{14} \\
	\gantttitlelist{1,...,7}{2} \\
	\ganttgroup{TC I}{6}{12} \\	
	\ganttmilestone{Entrega Proposta}{7} \ganttnewline	
	\ganttbar{Task 1}{6}{12} \\	
	\ganttbar{Task 2}{8}{8} \\
	\ganttbar{Task 3}{9}{9} \\
	\ganttbar{Task 4}{10}{10} \\
	\ganttbar{Task 5}{11}{11} \\
	\ganttbar{Task 6}{12}{12} \\	
	\ganttmilestone{Entrega Volume Final TC I}{12} \ganttnewline
	%\ganttlinkedgroup{Task 3}{2}{3}
	%\ganttlinkedbar{Task 3}{3}{7} \ganttnewline
	%\ganttmilestone{Milestone}{7} \ganttnewline
	%\ganttbar{Final Task}{8}{12}	
	\ganttlink{elem2}{elem8}
	\ganttlink{elem4}{elem6}
	\ganttlink{elem3}{elem4}
	\ganttlink{elem5}{elem6}	
	\ganttlink{elem6}{elem7}
\end{ganttchart}



