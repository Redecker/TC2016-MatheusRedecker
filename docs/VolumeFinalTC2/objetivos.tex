%!TEX root = volumeFinal.tex 

\chapter{\label{chap:ativ}Objetivos}

\frm[inline]{Este capítulo morre. Move os objetivos para a introdução, e comenta no que tu alcançaste na e como atingiste os objetivos na conclusão.}

Atualmente, há dificuldade na utilização de técnicas de IA para jogos em tempo real.
Isso ocorre pelo fato que as IAs precisam ser capazes de resolver tarefas de maneira rápida e que façam sentido dentro do jogo.
Geralmente, o espaço de estados dos jogos é enorme, isso faz com que não se tenha tempo de explorar todas as possibilidades de solução \cite{millington2009artificial}.

\section{Objetivo Geral}
O objetivo geral deste trabalho é implementar o algoritmo de planejamento AHTN \cite{ontanon2015adversarial} no jogo MicroRTS. O algoritmo de AHTN combina técnicas de HTN com o algoritmo de \textit{minimax serach}. Ao final da implementação, o objetivo é conseguir comparar o algoritmo com as demais técnicas presentes no MicroRTS.

\section{Objetivos Específicos}\label{obj:esp}
Os Objetivos são classificados em dois grupos, fundamentais e desejáveis. 
Os objetivos fundamentais representam a base do projeto.
Os objetivos desejáveis, representam objetivos destinados a tentar melhorar os resultados obtidos nos objetivos fundamentais. 

\subsection{Objetivos fundamentais} \label{sec:objfunda}

\begin{itemize}
	\item Implementar o algoritmo de AHTN no ambiente do jogo MicroRTS.
	\item Comparar os resultados obtidos com as técnicas presentes no MicroRTS.
	\item Enviar a implementação para o autor do MicroRTS como uma sugestão de IA para o jogo.
\end{itemize}

\subsection{Objetivos desejáveis} \label{sec:objdesej}
\begin{itemize}
	\item Integrar um algoritmo de aprendizado ao algoritmo de AHTN.
	\item Comprar os resultados obtidos com a implementação do AHTN.
	\item Escrever um artigo para um \textit{workshop} apresentando a unificação das técnicas de planejamento com aprendizado por reforço.
\end{itemize}

\section{Objetivos alcançados}

Este trabalho seguiu o plano das atividades proposto no Trabalho de Conclusão 1.
O objetivo de implementar o algoritmo de AHTN ao MicroRTS foi completo.
Com isso, foi possível comprar os resultados com as técnicas já implementas na plataforma.
O envio da implementação para o autor do MicroRTS será feita após a restruturação e organização do código para entrega do volume final.

Muito tempo foi gasto na união do MicroRTS com o JSHOP2. 
Com isso os objetivos desejáveis, descritos na Seção~\ref{sec:objdesej}, sofreram reajustes.
Não foi possível acoplar ao algoritmo de AHTN um algoritmo de \textit{Machine Learning}.
Por essa razão, também não foi possível comprar as duas abordagens.
O objetivo de escrever um artigo será reavaliado.
A reavaliação servirá para ver se este trabalho tem conteúdo para a publicação de um artigo.



