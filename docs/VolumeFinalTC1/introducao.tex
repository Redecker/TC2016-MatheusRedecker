%!TEX root = volumeFinal.tex 
\chapter{\label{chap:intro}Introdução}

Inteligência artificial (IA) é uma área em ciência da computação que tem como objetivo fazer com que o computador seja capaz de realizar tarefas que precisam ser pensadas, como é feito pelas pessoas. A IA possui uma área de aplicação em jogos, fazendo com que os computadores sejam capazes de jogar jogos como xadrez e jogo da velha. 
Os Jogos de computador muitas vezes não utilizam algoritmos de decisão autônoma ótima para simular o comportamento de jogadores; utilizando ao invés disso técnicas que passa uma ilusão de que as decisões estão sendo realizadas de forma autônomas, ou ainda utilizam técnicas que se aproveitam das informações provenientes do controle do jogo. Esse tipo de técnicas não pode ser caracterizado como uma técnica de IA totalmente autônoma \cite{millington2009artificial}. \frm[color=yellow]{Que tipo de truque? Tu queres dizer um comportamento scriptado que dá a ilusão de decisão autônoma?} \frm[color=yellow]{Explicar isto. Eu sei o que tu queres dizer, mas isto está longe de ser óbvio.}
%Por esse motivo existe uma discussão\frm{Existe uma discussão? Tu provoca uma certa curiosidade (que discussão é esta?!?) E depois corta o assunto no meio do parágrafo. Isto que eu quero dizer com a falta de conexão.} do que pode se caracteriza como uma IA em jogos \cite{millington2009artificial}. \frm[color=yellow]{Este parágrafo introdutório está meio raso no contexto do teu trabalho. Tem que ter uma conexão com o próximo parágrafo mais forte. Introduza jogos como uma área de aplicação interessante (por algum motivo) para IA.}

Jogos que utilizam técnicas de IA conseguem prover uma melhor interação entre o jogador e o jogo, tornando o jogo mais real e assim prendendo a atenção do jogador \cite{millington2009artificial}. \frm[color=yellow]{Só agora eu me liguei que isto é do mesmo paper do final do parágrafo anterior, mas isto não é NADA óbvio...}
Entretanto, os métodos utilizados, são geralmente mais simples do que os utilizados no meio acadêmico, pelo fato de que o tempo de resposta dos algoritmos é superior ao tempo que se tem para tomar uma ação ótima dentro do jogo e também pelo fato das ações geradas são mais previsíveis \cite{intelligence2003modern}.
Nos jogos de computador as reações dos oponentes devem ser quase que imediatas, por esse motivo técnicas que tentam explorar todo o espaço de estados possíveis de um jogo se tornam inviáveis para jogos com uma complexidade maior.
Por exemplo, no xadrez a quantidade aproximada de estados possíveis é de $10^{40}$, isso mostra que o poder de processamento para gerar, de maneira rápida, uma ação precisa ser alto \cite{millington2009artificial}. 
Então é difícil conseguir gerar uma ação ótima, em alguns casos, são gerados ações sub ótimas para que o tempo de resposta não seja muito alto \cite{intelligence2003modern}. 

Os jogos de estratégia em tempo real são jogos onde os jogadores devem construir uma base, conseguindo coletar recursos e traçar batalhas com o objetivo de derrotar os seus oponentes, jogos muito conhecidos como \textit{StarCraft}\footnote{http://us.battle.net/sc2/pt/} e \textit{Age of Empires}\footnote{http://www.ageofempires.com/} são exemplos de jogos desse gênero \cite{ontanon2013survey}.
Neste gênero de jogos é preciso decidir quais ações precisam ser tomadas, uma maneira de se fazer isso é utilizando técnicas de busca. As técnicas de busca almejam conseguir, uma técnica que consegue definir com essas ações é chamada de busca.
As técnicas de busca adversária conseguem determinar qual a próxima ação que deve ser tomada com o objetivo de ganhar o jogo. O problema dessas técnicas é que elas devem explorar pelo menos uma parte do espaço de estados, e com isso jogos que tenham uma grande quantidade de jogadas possíveis, esse tipo de técnica se tornar inviável, pelo tempo que é preciso para gerar uma ação \cite{ontanon2012experiments}. \frm[color=yellow]{Leia esta frase calmamente (mesmo com as minhas correções), e perceba que as frases antes e depois da vírgula não tem conexão nenhuma.} \frm[color=yellow]{Note que tu troca de RTS games para busca de supetão! É como eu dizer ``Em carros é preciso se mover, a reação química entre gasolina e oxigênio é chamada de combustão...''}

Na busca de mitigar as limitações de eficiência computacional de abordagens tradicionais de raciocínio em jogos, Ontañón e Buro propuseram o algoritmo chamado \textit{Adversarial Hierarchical Task Network (AHTN)} \cite{ontanon2015adversarial}. 
Neste algoritmo são combinadas técnicas de planejamento hierárquico com as de busca adversaria. 
Com o intuito de obter uma melhor performance na escolha das ações, propomos a utilização do algoritmo AHTN em conjunto com um algoritmo de aprendizado por reforço na plataforma MicroRTS\footnote{https://github.com/santiontanon/microrts}, que é um jogo de estratégia em tempo real. 
Com este trabalho pretendemos mostrar que o algoritmo de AHTN apresenta melhores resultados quando aplicado junto com técnicas de aprendizado por reforço. \frm[color=yellow]{Conecta com o próximo parágrafo, e junta os dois, tu não mudaste de assunto...}

Este documento é organizado da seguinte forma: No Capítulo \ref{chap:agentes} é apresentado o conceito de agentes e como podemos representar ele dentro dos diferentes problemas existentes. O Capítulo \ref{chap:busca} apresenta como podemos utilizar busca dentro do contexto de jogos, o Capítulo \ref{chap:planejamento} trata o conceito básico de planejamento para conseguir entender como o algoritmo de AHTN funciona. O Capítulo \ref{chap:aprendizado} apresenta técnicas de aprendizado de máquina que podem ser usadas em jogos. No Capítulo \ref{chap:obje} é proposto os objetivos para este trabalho, enquanto no Capítulo \ref{chap:ativ} é onde são definidas as atividades que serão realizadas em seguida, junto com a modelagem do problema.   
