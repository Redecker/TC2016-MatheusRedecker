%!TEX root = volumeFinal.tex 
\chapter{\label{chap:intro}Introdução}

Inteligencia artificial(IA) é uma área em ciência da computação que tem como objetivo fazer com que o computador seja capaz de realizar tarefas que precisam ser pensadas, como é feito pelas pessoas \cite{millington2009artificial}.  
A IA possui algumas áreas de aplicação, tais como: planejamento automatizado, jogos, robótica, tradução automática, entre outras \cite{intelligence2003modern}. 

As técnicas de IA utilizadas nos jogos são necessárias para conseguir uma melhor interação com o jogador, tornando o jogo mais real e assim prendendo a atenção do jogador \cite{millington2009artificial}. 
As técnicas utilizadas nos jogos, geralmente, são mais simples do que as que utilizadas no meio acadêmico, pelo fato de que o tempo de resposta dos algoritmos é superior ao tempo que se tem para tomar uma ação ótima dentro do jogo \cite{intelligence2003modern}. 
Nos jogos as reações devem ser quase que imediatas, para isso técnicas que tentam explorar todo o espaço de estados possíveis de um jogo se tornam inviáveis para jogos mais complexos.
Por exemplo, no xadrez a quantidade aproximada de estados possíveis é de $10^{40}$, isso mostra que o poder de processamento para gerar, de maneira rápida, uma ação precisa ser alto \cite{millington2009artificial}. Então é difícil conseguir gerar uma ação ótima, em alguns casos são gerados ações sub ótimas para que o tempo de resposta não seja muito alto \cite{intelligence2003modern}. 

%A busca é utilizada, dendo da IA, para achar a possível sequencia de ações que resolve um problema, considerando varias possibilidades de sequencia dessas ações. Os algoritmos de busca se diferenciam entre si na forma de escolher qual o próximo estado na busca pelo objetivo. Já a busca adversária é utilizada para a resolução de problemas de busca em modo competitivo. A busca adversária pressupõe que sempre o oponente irá realizar sempre a jogada que mais lhe beneficia, isso nem sempre acontece, seja porque o jogador é iniciante ou comete um erro. O ser humano consegue raciocinar para decidir as ações, mas nem sempre ele é perfeito nas escolhas das ações, ele consegue ser muito bom, mas o modo de pensar não o torna perfeito sempre. Planejamento é uma área da IA que busca a geração de planos de forma automática, parecido com a busca, utilizando técnicas para buscar a geração de um plano que satisfaça um objetivo. A utilização de técnicas de planejamento em jogos é uma tarefa difícil devido a sua grande quantidade de ações possíveis. Esse gênero de jogo possui um fator de ramificação muito grande, e cresce exponencialmente, com isso aplicar algoritmos de planejamento se torna uma tarefa não trivial \cite{intelligence2003modern}. 

Na busca de mitigar as limitações de eficiência computacional de abordagens tradicionais de raciocínio em jogos, Santiago Ontañón e Michael Buro propuseram o algoritmo chamado \textit{Adversarial Hierarchical Task Network (AHTN)} \cite{ontanon2015adversarial}. Neste algoritmo são combinadas técnicas de HTN com o algoritmo de busca adversaria \textit{minimax search}. 

Propomos a utilização do algoritmo AHTN combinado com uma técnica de aprendizado por reforço em um jogo de estrategia em tempo real combinando uma técnica de aprendizado por reforço. Com este trabalho pretendemos mostrar que o algoritmo de AHTN apresenta melhores resultados quando aplicado junto com técnicas de aprendizado por reforço.

Este documento está organizado da seguinte forma. No Capítulo \ref{chap:agentes} é apresentado o conceito de agentes e como podemos representar ele dentro dos diferentes problemas existentes. O Capítulo \ref{chap:busca} mostra como podemos utilizar busca dentro do contexto de jogos. O Capítulo \ref{chap:planejamento} apresenta o conceito básico de planejamento para conseguir entender como o algoritmo de AHTN funciona. No Capítulo \ref{chap:obje} é proposto os objetivos para este trabalho. O Capítulo \ref{chap:ativ} é onde são definidas as atividades que serão realizadas em seguida, junto com a modelagem do problema.   
