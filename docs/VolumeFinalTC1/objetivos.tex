%!TEX root = volumeFinal.tex 
\chapter{\label{chap:obje}Objetivos}

Atualmente, há uma dificuldade na utilização de técnicas de IA para jogos em tempo real. Isso ocorre pelo fato que a IA precisa ser capaz de resolver tarefas, do mundo real, de maneira rápida e satisfatória. Geralmente, o espaço de estados dos jogos é enorme, isso faz com que não se tenha tempo de explorar todas as possibilidades de solução \cite{millington2009artificial}.   

\section{Objetivo Geral}
O objetivo geral deste trabalho é implementar o algoritmo de planejamento AHTN \cite{ontanon2015adversarial} no jogo MicroRTS. O algoritmo de AHTN combina técnicas de HTN com o algoritmo de \textit{minimax serach}. Após implementado, integrar aprendizado por reforço com o objetivo de melhorar o desempenho do algoritmo. Ao final do trabalho, o objetivo é conseguir comparar resultados obtidos com os resultados \cite{ontanon2012experiments,hogg2010learning,ontanon2013survey}. 

\section{Objetivos Específicos}\label{obj:esp}
Os Objetivos são classificados em dois grupos, fundamentais e desejáveis. Os objetivos fundamentais representam a base do projeto, já os objetivos desejáveis, representam objetivos que devem ser realizados para uma melhor demonstração das técnicas apresentadas, e ainda para tentar melhorar os resultados obtidos nos objetivos fundamentais. 
 
\subsection{Objetivos fundamentais} 
%\frm[inline]{ Teus objetivos tem um problema de granularidade. Tu tens menos atividades do que objetivos, então acho que tu tens que repensar os objetivos (o QUE tu queres), e depois para cada objetivo, planejar uma ou mais atividades (COMO atingir os objetivos)}
\begin{itemize}
	\item Implementar o algoritmo de AHTN no ambiente do jogo MicroRTS.
	\item Comparar os resultados obtidos com as implementados já embutidas no MicroRTS.
\end{itemize}
 
\subsection{Objetivos desejáveis}
\begin{itemize}
	\item Integrar o algoritmo de \textit{Q-learning} a técnica de AHTN.
	\item Comprar os resultados obtidos com a implementação do AHTN puro.
	\item Enviar a implementação para o autor do MicroRTS como uma sugestão de IA para o jogo.
	\item Escrever um artigo para um \textit{workshop} mostrando a unificação das técnicas de planejamento com aprendizado por reforço.
\end{itemize}

