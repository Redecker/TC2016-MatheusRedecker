%!TEX root = exemplo.tex 
%% Matheus renomeia "exemplo.tex" para um nome mais descritivo (e muda a linha acima)
\chapter{\label{chap:conte}Contexto}

\todo[color=red,author=Felipe,inline]{Ok, aqui está bem genérico, agora começa a projetar parágrafo por parágrafo do material que tu já leste (bullets para cada parágrafo), e me manda depois que tu completares o texto. Deixa as bullets no topo para eu entender o teu raciocínio}

\cite{meneguzzi2015planning}
\cite{ontanon2015adversarial}
\cite{ontanon2012experiments}
\cite{intelligence2003modern}
\cite{buro2003real}
\cite{hogg2010learning} \msr{apenas para aparecem as referencias}

\section{Planejamento} 
-Overview de planejamento \\
-definição \\
-conceitos \\
-metodos \\
-aplicação no mundo real

\subsection{HTN} 
-como se encaixa em planejamento \\ 
-definição \\
-conceitos \\

\subsection{AHTN} 
-como se encaixa em planejamento e HTN \\
-definição \\
-conceitos

\section{Aprendizado} 
-overview de aprendizado

\subsection{Aprendizado de Máquina} 
- overview de aprendizado de maquina \\
- definição \\
- conceitos
\subsection{Aprendizado por Reforço} 
- como se encaixa dentro de aprendizado de maquina \\
- definição \\
- conceitos

\section{Jogos Real-time Strategy(RTS)} 
Explicação do que é jogos RTS \\
ligação deles com atividades do mundo real \\
exemplos de jogos RTS 

\subsection{MicroRTS} 
explicação de como funciona, para que foi feito. 
